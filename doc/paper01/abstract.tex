%%-*-latex-*-

\begin{abstract}
  This paper explores the development of a system for helping students
  to learn functional programming. A recurrent issue is to grasp the
  concept of recursion, in contrast with iteration and this is
  particularly relevant with pure functional languages, like \Erlang,
  where function calls are the only control mechanism available.
  
  The situation may seem paradoxical, since recursion is being taught
  in mathematics courses in high-school and tail-recursion is
  theoretically equivalent to iteration.
  
  Based on our experience and review of the literature, we propose
  several hypotheses to understand the problem and put some to the
  test by designing an open-source interactive interface based on a
  tangible block-world with augmented reality and software
  feedback. After using this application, the students were able to
  transfer their training to directly write general recursive programs
  in \Erlang.
\end{abstract}
