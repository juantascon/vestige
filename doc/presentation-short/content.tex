\section{Goal}

\begin{frame}{Goal}
  Ease learning functional programming by means of an interactive
  interface based on a tangible block-world with augmented reality
  and software feedback.
  \vfill
  \textsf{Tags}: functional programming, tangible user-interface,
  block world, augmented reality, software feedback
\end{frame}

\section{Functional Programming}

  \subsection{Definition}
    \begin{frame}{Functional programming}
      \emph{Functional programming} is a programming style that relies mostly on
      mathematical functions and immutable data.
      \vfill
      \pause
      \begin{itemize}
        \item No arrays
        \item No loops
        \item No variable asignations
      \end{itemize}
    \end{frame}

    \subsection{Difficulties}
      \begin{frame}{Difficulties}
        \begin{itemize}
          \pause \item If students are first taught programming in
          an \emph{imperative language}, they may be encouraged to rely mostly on \emph{iteration}.
          \pause \item When using strings and arrays, loops are the ``natural'' way of thinking a solution.
          \pause \item Base cases are hard to identify.
          \pause \item Students usually think in a ``loop'' model of recursion
          and not a "copies" model.
          \pause \item It is hard to trace when an instantiation starts and
          ends, before its preceding instantiation ends.
        \end{itemize}
      \end{frame}

\section{Demo}
  \subsection{Example}
    \begin{frame}{Example}
      \emph{Example:}
      \lstinputlisting[language=erlang,breaklines=true]{reverse.el}
    \end{frame}
