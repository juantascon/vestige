%%-*-latex-*-

\section{Methodology}

\noindent\textbf{System overview.} The application challenges the
student to solve a problem by interacting with stacks of blocks placed
on a table. A complete session is best understood as a series of
scenes, that is, snapshots of the board, which abstracts the execution
trace of an \erlang function on a given input, i.e., the initial
setting of the board. Markers are signs that are captured and
recognized by the system from the physical scene. They are
specifically designed to be easily understood by the \artoolkit (which
is a back\hyp{}end for \osgart) and are stuck on each block. They come
in three kinds: \textsl{Item}, \textsl{Stack} and \textsl{Switch}
(whose occultation by the hand triggers a snapshot). For each
recognized marker, a number, a variable, a picture may be superimposed
to the video feed and sent to a monitor. This 3D\hyp{}capable AR
feature is supported by another back\hyp{}end of \osgart called \osg.

\medskip

\noindent\textbf{Interactions.} The stack data structure can be
modified only by two simple operations: pop and push. These are the
minimum set of operations required to allow us to change the data in
any way we wish. However, in our system it is necessary to recognize
the block source and destination, that is, a stack, the table (it then
represents a value which is not a stack) or outside/inside the scene,
so the resulting set of operations is the following:
\begin{itemize}

  \item \textsl{Pop:} moving the top block from a stack onto the
    table;

  \item \textsl{Push}: moving a block from the table onto the top of a
    stack (dual of \textsl{Pop});

  \item \textsl{Pop-Push:} moving a top block from a stack on top of
    another stack (logical composition of \textsl{Pop} and
    \textsl{Push});

  \item \textsl{Discard:} moving a block or stack out of the scene,
    regardless of where it is located;

  \item \textsl{Create:} moving a block or a stack from outside the
    scene onto the table or the top of a stack (dual of
    \textsl{Discard}).
    
\end{itemize}
Changing a scene by any other means or by combining at once two or
more of these operations is considered invalid. These situations are
detected by the application which then stops the session with an error
message explaining what happened and prompts the student to restart
from scratch by placing the blocks back into a valid initial scene.

The system offers two execution modes: one called \emph{free},
allowing the student to freely, but validly, move blocks and another
one called \emph{supervised} in which each operation is checked to
conform to a possible solution, otherwise it is rejected and signalled
by a visual cue.

Adding new problems requires validating initial and final scenes and
dynamically creating \emph{positive scenarios}, that is, admissible
execution traces for different possible \erlang functions
corresponding to the problem. There are currently five problems
defined: stack reversal, joining two stacks, removing all the
occurrences of a given block in a stack, removing all consecutively
repeated blocks and sorting by insertion. The last two exercises are
more than structurally recursive because their assume some property on
the denotations of the blocks, equality and a total order,
respectively.

\bigskip

\noindent\textbf{A session.} Through the interface, blocks augmented
in yellow denote the base of stacks and light blue blocks represent
items belonging to a stack. The block with a camera icon projected
upon stands for the switch that, when hidden, triggers a scene
snapshot. A session ends with the system superimposing a
congratulation message if the student provides a correct and validly
obtained answer. In the case of an invalid action, the system displays
a message indicating the cause of the mistake, the blocks involved in
the invalid move will be augmented with red and the student will have
to restart the session from the beginning.

To illustrate how the interface works, we are going to step through a
sample session of the exercise consisting in joining two stacks, that
is, forming a new stack by moving all the items of one stack onto the
top of another, while retaining their original order. Textually, in
\erlang, this means that, given the stacks \texttt{P} and \texttt{Q},
with \texttt{P} denoting \texttt{[1,2]} (the top is \texttt{1} and the
bottom \texttt{2}) and \texttt{Q} denoting \texttt{[3,4]}, the joining
of \texttt{P} and \texttt{Q} is \texttt{[1,2,3,4]}. The user starts
the session by setting two stacks on the table, representing the input
arguments \texttt{P} (on the left) and \texttt{Q} (on the right) in
\erlang, i.e., \texttt{join(P,Q)} is the function call to
compute. Some random integers are superimposed on the blocks (they are
not random in the coming snapshots, so it is easier to follow the
transformations). This scene can be seen in
Figure~\ref{fig:head1}. 
\begin{figure}[!h]
\centering
\subfloat[\texttt{join([1,2],[3,4])}\label{fig:head1}]{
\includegraphics[scale=1.0]{img/step0.png}
}
\quad
\subfloat[\texttt{join([],[1,2],[3,4])}\label{fig:body1}]{
\includegraphics[scale=1.0]{img/step1.png}
}
\caption{\texttt{\small join(P,Q) -> join([],P,Q)}
\label{fig:clause1}}
\end{figure}
The change of scene is denoted in \erlang by an arrow \texttt{->} at
the left of which is a pattern for the current scene and at the right
is the next scene in terms of the previous one. The final result
\texttt{[1,2,3,4]} is shown on the display and the user is prompted to
introduce from outside the scene an empty stack at the leftmost side
of the scene, which will be used as a temporary accumulator (operation
\textsl{Create}). This new scene is in Figure~\ref{fig:body1} and the
\erlang piece of code whose instanciation applies here is {\small
\begin{verbatim}
join(P,Q) -> join([],P,Q).
\end{verbatim}
}
\noindent 
where \texttt{[]} represents the empty stack. The user is now expected
to move top blocks from the middle stack (\texttt{P}) to the left
stack (operation \textsl{Pop\hyp{}Push}), as shown in
Figure~\ref{fig:clause2}.
\begin{figure}[!h]
\centering
\subfloat[\texttt{join([],[1,2],[3,4])}]{
\includegraphics[scale=1.0]{img/step1.png}
}
\quad
\subfloat[\texttt{join([1],[2],[3,4])}]{
\includegraphics[scale=1.0]{img/step2.png}
}
\caption{\texttt{\small join(A,[I|P],Q) -> join([I|A],P,Q)}
\label{fig:clause2}}
\end{figure}
This transformation is captured by the \erlang clause
{\small
\begin{verbatim}
join(A,[I|P],Q) -> join([I|A],P,Q);
\end{verbatim}
}
\noindent In \erlang, \texttt{[I|P]} is a pattern which matches any
non\hyp{}empty stack whose top block is named \texttt{I} and the
remaining stack \texttt{P}. After repeating this operation one more
time on our example, the scene is as shown in Figure~\ref{fig:head3}.
\begin{figure}[!h]
\centering
\subfloat[\texttt{join([2,1],[],[3,4])}\label{fig:head3}]{
\includegraphics[scale=1.0]{img/step3.png}
}
\quad
\subfloat[\texttt{join([1],[],[2,3,4])}\label{fig:body3}]{
\includegraphics[scale=1.0]{img/step4.png}
}
\caption{\texttt{\small join([I|A],[],Q) -> join(A,[],[I|Q])}
\label{fig:clause3}}
\end{figure}
It is matched by the pattern \texttt{join([I|A],[],Q)}, meaning ``left
stack with top block \texttt{I} and sub\hyp{}stack \texttt{A}, middle
stack empty and right stack \texttt{Q} unchanged.'' The next phase
consists in moving the blocks from the leftmost stack to the rightmost
(operation \textsl{Pop\hyp{}Push}). This is shown in
Figure~\ref{fig:clause3}, which corresponds to an instance of the
piece of source code 
{\small
\begin{verbatim}
join([I|A],[],Q) -> join(A,[],[I|Q]);
\end{verbatim}
}
\noindent After repeating this operation once more on our example, the
left stack becomes empty and the corresponding scene is seen in
Figure~\ref{fig:head4}. 
\begin{figure}[!h]
\centering
\subfloat[\texttt{join([],[],[1,2,3,4])}\label{fig:head4}]{
\includegraphics[scale=1.0]{img/step5.png}
}
\quad
\subfloat[\texttt{[2,3,4]}\label{fig:body4}]{
\includegraphics[scale=1.0]{img/step6.png}
}
\caption{\texttt{\small join([],[],Q) -> Q}}
\label{fig:clause4}
\end{figure}
The \erlang pattern for it is \texttt{join([],[],Q)}. The result is
finally reached by keeping only the non\hyp{}empty stack on the scene
(operation \textsl{Discard} twice), as shown in
Figure~\ref{fig:clause4}. This is expressed by the clause
{\small
\begin{verbatim}
join([],[],Q) -> Q.
\end{verbatim}
} 
The student is not presented the \erlang at this stage, we wanted to
show the analogy between the AR scene and the textual program, which
the student in a later session will be asked to write directly. Here,
the complete \erlang program of this session was as follows (module
declarations omitted):
{\small
\begin{verbatim}
join(          P,Q) -> join(   [], P,    Q).
join(    A,[I|P],Q) -> join([I|A], P,    Q);
join([I|A],   [],Q) -> join(    A,[],[I|Q]);
join(   [],   [],Q) -> Q.
\end{verbatim}
}

\medskip

\noindent\textbf{Scope and limitations.} From the standpoint of
programming expressivity, the system is limited to functions based on
structural recursion on stacks and constant values. This is a
consequence of using a TUI, so adding new interactions would require
technologies that are not widely spread, for example gesture
recognition, motion tracking, voice recognition, etc. On the other
hand, with the current system, it just as easy to use rectangular
pieces of paper on a table instead of blocks, making it as portable as
the laptop with a webcam running it. The set of definable function is
also restricted to tail\hyp{}recursivity because representing the
control stack would require a special stack growing top\hyp{}down, so
the order of the instances of the control contexts is preserved. Since
the system is intended to be used by novices as a temporary tool to
understand simple cases of recursion, instead of as a general
programming environment, this is not an impediment.

\medskip

\noindent\textbf{Future experiments.} The first phase is made up of
three stages, which are repeated. In a first stage, the students will
be asked to build some stacks and randomly chosen integers will be
displayed on the blocks, one per block. The expected result will in
turn be shown on the screen. The student may select either the free or
supervised execution mode and try to reach the goal. After, two more
random examples for the same problem will be proposed. This stage
relies only on \emph{concrete examples}, as the one illustrated above
with \texttt{join}. It presents some similarities with the framework
of \emph{programming by example}, except that the goal is not to teach
the computer how to program but the other way round.

In the second stage, another random example for the same problem is
displayed but only the top of the stacks will be visible. This is
meant to induce in the learner the understanding that the exercise can
be solved without relying on global knowledge and that
\emph{information hiding} actually helps to focus the attention on the
smallest part of the data which is needed to make one more little step
towards the result. Each time a \textsl{Push\hyp{}Pop} operation is
performed, for instance, only the topmost blocks are consistently
shown.

After three runs like this, three other examples of the same problem
are offered, but instead of integers superimposed on the block
markers, \emph{variables} are. This last stage aims at giving rise to
\emph{abstraction} and prepares the transfer to textual programming.

When the last stage is over, another problem is submitted to the
student etc. until the interface hopefully becomes useless and the
direct programming of the \erlang functions corresponding to the
previous exercises is attempted. In the second phase, the professor
shows the analogy between, on the one hand, the stacks of blocks and
the valid operations learnt by means of the interface and, on the
other hand, the \erlang syntax for lists and the \erlang semantics. We
expect to measure a statistically significant transfer of training for
students who used the system in comparison with students who did not.

