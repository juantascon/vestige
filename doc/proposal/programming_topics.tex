%%-*-latex-*-

\section{Programming Topics}

At the end of the system\hyp{}student interaction she should be able
to write and understand short and simple \Erlang programs operating on
integers, atoms (an enumerated type) and lists. We classified the
algorithms based on the concepts they embody as follows.

\begin{description}

  \item[Purely structural] These programs rely solely on the structure
    of the data, not its denotation (abstract value). Here, that
    translates as traversals and/or transformation of lists,
    regardless of the nature of the items. Examples: extracting the
    last or penultimate element of a list, joining two lists,
    reversing a list.

  \item[Structural with equality] These programs are structural but
    also require the equality operator, which can be implicit in
    \Erlang as non\hyp{}linear patterns, where a variable occurs
    several times. Examples: copying a list without the first
    occurrence of a given element, ditto but filtering out all the
    occurrences, ditto but avoiding the last occurrence, compressing a
    list by removing repetitive and consecutive items in a list.

  \item [Structural with nesting] These programs work on lists whose
    items are in turn lists. (Two levels of such an embedding are
    enough for most interesting cases.). Examples: making the list of
    the repeated consecutive items in a list), flattening a list so
    the result is a list containing the same original non\hyp{}list
    items but no lists.

  \item[Structural with partial or total ordering] These programs rely
    on some total order on values, like sorting by insertion.

  \item [Structural with arithmetic] They require the usual
    arithmetic operations \((+)\), \((-)\), \((\times)\), \((/)\)
    etc). Examples: duplicating \(n\) times the items of a list in the
    same input order, the factorial function.

  \item [Structural with higher\hyp{}order] These programs make use of
    functions as arguments. The classical example is sorting using a
    caller\hyp{}supplied function to compare any two elements.

\end{description}
