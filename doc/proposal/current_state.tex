\section{Current State}

Lists are algorithmically stacks, that is, linear data structure
allowing only two basic operations: \emph{push} and item on the top of
the stack or \emph{pop} the topmost item of the stack. We have been
investigating the best visual representation of stacks.

Our initial ideas included the use of a kind of domino chips
distributed across a board or surface, in this case a list would be
represented by a series of chips attached to each other, setting up a
row or a column.

Another option involved some fancy devices that would give us the
ability to attach objects in a given surface using magnetic fields,
but we abandoned the idea due to lack of information on previous
devices.

We finally set up our mind on a block\hyp{}world representation to
model lists. A block represents an element and a sequence of blocks
represents a list. The initial intuition is that real stacks of cubes
would be subjected to gravity, thus forcing the student to only handle
the topmost item of each list, as the model mandates. (Manipulating
another cube would lead the whole stack to collapse.) We then decided
to build a virtual block\hyp{}world which would analogous to a real
block\hyp{}world, more precisely, we will use technologies from the
field of \emph{Augmented Reality} (AR). This is a real time process
that creates the illusion of additional objects in a real life scene,
making the user experiment with an alternate reality as a mix between
real life and virtual reality. It is a four\hyp{}step loop procedure:
\begin{enumerate}
	
	\item \emph{Real life image acquisition.} This could be done using
      cameras for real\hyp{}time experience or using previously taken
      videos for testing and developing purposes.
	
	\item \emph{Locating the new objects target position and
      inclination.} This is the more complex process, which consists
      in recognising predefined patterns by means of several
      approaches such as the use of markers, body parts recognition,
      pre\hyp{}modelling of objects, etc.
	
	\item \emph{Overlapping the new objects (2D or 3D) on the initial
      video.} This is a very simple task, given a system of
      coordinates and and inclination transformation matrix.
	
	\item \emph{Displaying the resulting image.} The resulting image
      is the combination of the initial image and the new objects.
	
\end{enumerate}
We have been working and testing some of the most popular
implementations of Augmented Reality, first of all \ARtoolkit
(\emph{Augmented Reality toolkit}). The main strengths of \ARtoolkit
are, on the one hand, the royalty\hyp{}free licence (GNU General
Public License) which allowed us to receive feedback in the form of
patches, comments, bug reports, etc. from a community of users and
developers around the world, and, on the other hand, the simplicity of
the mark detection process (a mark is a image\hyp{}description couple
processed by \ARtoolkit to recognise targets position and inclination
on the initial image).

Some troubles emerged while testing \ARtoolkit on Linux platforms,
specially in the image handling, acquisition and re\hyp{}displaying
procedures, because the video back\hyp{}ends were outdated and
unstable. Efforts spent searching for a fix lead us to a stable and
portable development branch of \ARtoolkit. It is a common occurrence
in free software that the development branch is very stable.) So we
were able to run the demos on Linux systems without any troubles,
while retaining the portability to the Windows operating system.

At this point we discovered \Osgart, which is a library that mixes in
an efficient and simple \emph{Application Programing Interface} (API)
the four AR steps, using plug\hyp{}ins. The default \Osgart
plug\hyp{}ins are: \ARtoolkit for the image acquisition and target
detection and \Openscenegraph for the image overlapping and display
(it is a toolkit to display, order and adapt images and 3D objects
scenes). We did not use systematically this tool because it was not
necessary for our first tests, but we plan to give it serious
consideration in future stages.

Special AR glasses have become popular in AR applications, enabling
both recording of video images (with integrated cameras) and
displaying (on the lens) images from the real world. Optionally, the
image could be preprocessed and modified before displaying. This
process is made fast enough in order to not be detected by a human
giving to the user an unique real time augmented reality experience.
