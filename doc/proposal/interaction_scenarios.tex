%-*-latex-*-

\section{Interaction and Scenarios}

Another aspect of our research consists in exploring different
interface technologies, as augmented reality, several interactivity
devices, programming by demonstration, etc. The main concern here is
to find the best fitted interactive process in order to get the
student to learn the concepts to be programmed. We could use different
technologies to teach different concepts, for instance we could use
augmented reality to teach stacks and queues but use haptic devices to
teach trees. Once the interface capabilities have been explored, the
research can move forward. Let us consider a likely scenario of
interaction:
\begin{enumerate}

  \item Following the classification set previously on our \Erlang
    functions, a function is chosen by the system.

  \item The didactic goals are, as expected, to help the student to
    write \Erlang code and help her when she is erring. An example
    here is a trace of execution of the chosen function, i.e., some
    input is chosen (perhaps randomly, perhaps not), and the sequence
    of steps of the execution is visually shown to the student as
    blocks moving around in stacks, until the final state is reached.

  \item Then the system proposes an initial state and the student is
    expected to move the blocks herself to find the result. This
    supposes an interaction with the virtual representation. This is
    where the scenarios are used, in order to detect a
    misunderstanding. In case this happens, another example can be
    completely demonstrated by the system, or a simpler example can be
    proposed, showing clearly the wrong assumption to the student when
    she tries to solve it.

  \item When the system is convinced that the student has grasped the
    concept represented in \Erlang by a (hidden) given function on
    stacks (that is, lists), the next level is to introduce
    \emph{abstraction}. Until now, everything involved in the
    interaction was concrete: a block is an atomic piece of data,
    identified with a specific mark (a letter or a number on
    it). Abstraction consists in understanding an infinite number of
    examples. The process proposed here is based on induction from a
    series of examples. Notationally, the key to abstraction is the
    variable, which is a symbol (like a name) denoting an infinite
    number of things. In the example of removing the bottom-most block
    in a stack, the number of blocks on top of it does not really
    makes a difference, because, if there are more than one the
    operations consists simply in repeating the same gesture. The
    system would replay the previous examples but making use of
    variables.

  \item The last stage consists in demonstrating the \Erlang code,
    after several exercises, the student will be prompted to write the
    \Erlang code by herself.

\end{enumerate}
Therefore, there are (1)~the HCI, (2)~the code database with
scenarios, (3)~a protocol of interaction between the user and the
system through the interface, (4)~a didactic methodology.
