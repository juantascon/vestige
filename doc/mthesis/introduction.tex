%%-*-latex-*-
%*******************************************************************************
\section{Introduction}
%% \addcontentsline{toc}{section}{Introduction}
%*******************************************************************************

Functional programming is a programming style that relies mostly on
mathematical functions and immutable data. On the one hand, a function
is mathematical when its output depends only on its input and, on the
other hand, immutable data is assigned only once. As a consequence,
this style discourages the use of concepts such as \emph{arrays},
\emph{loops} or \emph{side\hyp{}effects}. This carries along several
benefits such as \emph{concurrency}, easier \emph{unitary tests} and
safe \emph{multi\hyp{}threading} and also, \emph{recursion} becomes
the only control\hyp{}flow mechanism.

Recursion is the syntactic property of a definition to be
self\hyp{}referential and it is introduced in high\hyp{}school
mathematics as \emph{progressions} and \emph{proofs by induction}.
However, when it is taught in college, most students present great
difficulties to understand it. Several aspects have been studied to
solve this inconvenient including: cognitive sciences, the strong
mathematical background required, and the side\hyp{}effects of
\emph{iteration} as a pervasive concept in most computer science
curricula. This study proposes and develops a new approach to abate
these difficulties. The main goal is to ease learning recursion and
functional programming by means of an interactive interface based on a
tangible block\hyp{}world with augmented reality and software
feedback.

Productivity is the main measure to compare Human\hyp{}Computer
Interfaces. A clear demonstration of this are graphic tablets. They
are usually recognized as offering a very natural way to create
computer graphics, thus, providing a better interface in contrast to
traditional devices such as mice or keyboards. This \emph{naturalness}
of use underlies a significant increase in productivity, turning them
into very attractive devices for designers.

In the early 90s Tangible User Interfaces (TUI) were introduced in
the field of Human\hyp{}Computer interaction. In recent years,
several approaches have tried to explore the wide variety of possible
interaction mechanisms offered by TUIs. As a result, devices such as
video game controllers, haptics or virtual reality helmets can be
purchased in a local market. Everything seems to indicate that TUIs
have good potential as learning tools. The possible learning benefits
are related to the use of physical materials and 3D forms providing
the user with a more intuitive interface.

This dissertation is structured as follows.
The first chapter tries to bundle certain studies relevant to the
problems on learning recursion, interfaces for functional programming
training and the potential of TUIs on learning.
The second chapter describes in more depth the methodology used,
including the related concepts, a proposed didactic process and
the interaction through the interface, as well as a technical
overview of the system.
The third chapter details the selected test cases, which are five
common recursion problems implemented within the interface.
The fourth chapter explains the conducted experiments and their
results.
Finally, the last chapter states the conclusion and the future work.
