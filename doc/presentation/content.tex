% **
\section{Goal}
% **

\begin{frame}{Goal}
To help students on the process of learning functional programming with an
interactive application based on a tangible block world boosted by augmented
reality and software feedback.
\footnote{tags: functional programming, tangible user interface, block world,
AR, software feedback}
\end{frame}



% **
\section{Functional Programming}
% **

\subsection {Definition}
\begin{frame} {Definition}
Functional programming is a programming style that focuses on computation as the
evaluation of mathematical functions and avoids mutable data
\cite{wikipedia_functional_programming}.
\end{frame}

\subsection {Benefits}
\begin{frame} {Benefits}
\begin{description}
\pause \item [Unit Testing] the result of a function depends only on its
arguments
\pause \item [Debugging] if a return value of a function is wrong, it is always
wrong, regardless of what code you execute before running the function
\pause \item [Concurrency] a functional program is ready for implicit
concurrency without any further modifications. e.g. \emph{ f(g(),h()) }
\pause \item [Hot Code Deployment]
\pause \item [Machine Assisted Proofs and Optimizations]
\end{description}
\end{frame}

\subsection {Difficulties}
\begin{frame} {Difficulties}
\begin{itemize}
\pause \item Base cases are hard to identify \cite{haberman02}
\pause \item Students usually think in a "looping" model of recursion and not a "copies" model \cite{kahney83}
\pause \item It is hard to trace, an instantiation starts and ends, before its preceding instantiation ends \cite{ginat99}
\end{itemize}
\end{frame}



% **
\section{HCI}
% **

\subsection {Tangible User Interface}
\begin{frame} {Tangible User Interface}
\begin{itemize}
\pause \item Manipulation of physical objects can be of particular educational benefit \cite{montessori12}
\pause \item There have been positive precedent of TUIs on the field of learning programming \cite{fernaeus06} \cite{suzuki95}
\pause \item Physical manipulation might support more effective or more natural learning \cite{sluis04} \cite{terrenghi05} \cite{zuckerman05}  
\end{itemize}
\end{frame}

\subsection {Block World}
\begin{frame} {Block World}
\begin{itemize}
\pause \item Simplicity (blocks on top of other blocks)
\pause \item List is the basic data structure in functional programming languages
\pause \item Gravity
\end{itemize} 
\end{frame}
%TODO: imagen: [img: http://www.cksinfo.com/clipart/toys/abc-blocks.png]

\subsection {Augmented Reality}
\begin{frame} {Augmented Reality}
AR gives us the capability to alter the way the user visualize its physical
environment by adding virtual objects used to hide or show extra information.
\end{frame}

\subsection {Software Feedback}
\begin{frame} {Software Feedback}
Errors, tips and guides will be popped out on the block-world using the
augmented reality layer, helping the student through the process of solving the
proposed problem.
\end{frame}

\subsection {Demo}
\begin{frame}[fragile] {Demo}
Reverse a list's items (Erlang):
\begin{lstlisting}[language=erlang]
reverse(       [ ],Output) -> Output;
reverse([Top|Rest],Output) -> reverse(Rest,[Top|Output]).
\end{lstlisting}
\end{frame}




% **
\section {Issues}
% **

\begin{frame} {Issues}
\begin{description}
\pause \item [Conceptual] by focusing on students to overcome the difficulties
when learning functional programming
\pause \item [Technical] by developing a multi-platform open source software
application using unstable libraries
\pause \item [Interactive] by implementing the best fit software feedback
\pause \item [Practical] by testing the application on real programming students
to analyze the effectiveness of the method
\end{description}
\end{frame}



% **
\section {Proposed Solution}
% **

\subsection {Development phase}
\begin{frame} {Development phase}
\begin{itemize}
\pause \item Open source (GPL)
\pause \item CPP
\pause \item OSGArt
\end{itemize}
\end{frame}

\subsection {Test phase}
\begin{frame} {Test phase}
\begin{itemize}
\pause \item Problems design
\pause \item System deployment
\pause \item Measure Transfer-of-Training
\end{itemize}
\end{frame}



% **
\section {Future Work}
% **

\begin{frame}{Future Work}
\begin{itemize}
\pause \item Finish the base development (ASAP)
\pause \item Choose the tests problems wisely
\pause \item Run the tests on the subjects
\pause \item Write paper and thesis
\pause \item Graduate
\end{itemize}
\end{frame}
