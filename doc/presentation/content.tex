%*******************************************************************************
\section{Introduction}
%*******************************************************************************

\begin{frame}{Goal}
  Ease learning functional programming by means of an interactive
  interface based on a tangible block-world with augmented reality
  and software feedback.
  \vskip2ex
  \textsf{\bf Keywords:} \em{functional programming, tangible
    user-interface, block world, augmented reality, software
    feedback}
\end{frame}

\begin{frame}{Definition}
  \emph{Functional programming} is a programming style that relies
  mostly on mathematical functions and immutable data.
  \vskip2ex
  \begin{itemize}
  \item No arrays
  \item No loops
  \item No variable assignations
  \end{itemize}
\end{frame}

%*******************************************************************************
\section{Related Works}
%*******************************************************************************

\begin{frame}{Recursion Issues}
  \begin{itemize}
  \item Cognitive load theory: limited work memory.\footnote{The effect of dynamic copies model in teaching recursive programming: Chen(1998)}.
  \item Base cases are hard to identify\footnote{The case of bad cases: Haberman \& Averbuch(2002)}.
  \item Students usually think in a ``loop'' model of
    recursion and not a ``copies'' model\footnote{Mental models of recursion revisited: Sanders, Galpin \& Gotschi(2006)}.
  \end{itemize}
\end{frame}

\begin{frame} {TUIs used at Learning}
  The use of physical materials per se could help on the
  process of learning~\footnote{Do tangible interfaces enhance
    learning: Marshall(2007)}.
  \vskip2ex
  
  \begin{itemize}
  \item Using material change the nature of the knowledge
    because perception and cognition are closely interlinked.
  \item 3D forms might be perceived and understood more readily
    through haptic or other tangible representations than
    through visual representation alone.
  \item In young children manipulation of concrete
    physical objects helps to support and develop thinking.
  \end{itemize}
\end{frame}

%*******************************************************************************
\section{Methodology}
%*******************************************************************************

\begin{frame}{Interaction}
  \begin{description}
  \item[What?] A block-world TUI with AR to provide feedback.
  \item[How?] The learner must solve a programming exercise by
    reorganizing a set of blocks forming stacks (lists). Blocks'
    movements are guided and limited by gravity.
  \end{description}
\end{frame}

\begin{frame}{Elements}
  \begin{description}
  \item[Board:] The captured surface.
  \item[Blocks:] Any physical element above the board.
    (Item, List, Switch).
  \item[Actions:] A movement of blocks (Pop, Push, Pop-Push,
    Create, Discard).
  \item[Scenes:] An arrangements of blocks.
  \end{description}
\end{frame}

\begin{frame}{Abstraction (1/2)}
  When an action is triggered two scenes are identified:
  \vskip2ex
  \begin{description}
  \item[Previous scene:] Before the action is executed.
  \item[New scene:] After the action is executed.
  \end{description}
  \vskip4ex
  Erlang functions($f(x)=y$) are then given as the new scene
  ($y$) in terms of the previous scene($f(x)$).
\end{frame}

\begin{frame}{Abstraction (2/2)}
  \emph{Reverse:}
  
  \begin{enumerate}
  \item $\vestigefunction{rev(L)} {create} {rev(L,[])}$
  \item $\vestigefunction{rev([T|R],O)} {poppush} {rev(R,[T|O])}$
  \item $\vestigefunction{rev([\;],O)} {discard} {O}$
  \end{enumerate}
\end{frame}


%*******************************************************************************
\section{Future Work}
%*******************************************************************************

\begin{frame}{Future Work}
    \begin{itemize}
    \item The software still needs a bit of love.
    \item Write thesis.
    \end{itemize}
\end{frame}


\begin{frame}{Q/A?}
  \krtext{감사합니다}.
  \vskip3ex
  Q/A?
\end{frame}
